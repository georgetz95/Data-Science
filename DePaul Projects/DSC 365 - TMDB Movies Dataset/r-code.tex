\lstset{style=rstyle}

\begin{lstlisting}
library(dplyr)
library(lubridate)
library(ggplot2)
library(scales)

# For inflation adjustments
library(priceR)

# For my Sankey
library(highcharter)
library(htmlwidgets)
library(webshot)

# For pivot_longer()
library(tidyverse)

# For linear correlation annotation on scatterplots
library(ggpubr)

# A couple of my libraries conflict in a way that forced me to specify 
# "dplyr::" for aggregation functions.
\end{lstlisting}

\begin{lstlisting}
# Loading and preprocessing data.
movie_df_full <- read.csv("movies_complete.csv")
head(movie_df_full)
\end{lstlisting}

\begin{lstlisting}
### Creating a subset that primarily contains the attributes I'm interested in
# First, selecting specific attributes
mov_df <- movie_df_full %>% select(title, release_date, revenue, budget, popularity, vote_count, vote_average, runtime, original_language, PrimaryGenre, PrimaryProductionCompany, PrimaryProductionCountry )
\end{lstlisting}

\begin{lstlisting}
### Creating *release month* and *release year* from existing attribute *release date*
# From the "release_date" attribute, I create several attributes:
# release_month, release_year, and release_mon_yr .

mov_df <- mov_df %>%
  mutate(release_date = as.POSIXct(strptime(release_date, '%Y-%m-%d')),
         release_month = month(release_date),
         release_year = year(release_date))

head(mov_df)

mov_df$release_mon_yr = format(as.Date(mov_df$release_date), "%m-%Y")

# head(mov_df, 10)
 mov_df %>%
   count(release_year)
\end{lstlisting}

\begin{lstlisting}
### Removing data older than 1996 and newer than 2015
# 2015 is the most recent year of complete data in this set.

mov_df <- mov_df %>%
  filter(release_year <= 2015 & release_year >= 2001)

 mov_df %>%
   count(release_year)
\end{lstlisting}

\begin{lstlisting}
### Removing some records with invalid *revenue* and *budget* values
# The dataset contains values of zero for revenue and budget that strike us as
# invalid. Removing tuples that show those values.

nrow(mov_df)

mov_df <- mov_df %>%
  filter(revenue != 0 & budget != 0)

nrow(mov_df)
\end{lstlisting}

\begin{lstlisting}
### Defining the function for inflation adjustment
country <- "US"
inflation_dataframe <- retrieve_inflation_data(country)
countries_dataframe <- show_countries()

# 2022 is the most recent year that this function works for.
adjust_for_inflation(100, 2016, country, 2022,
                     inflation_dataframe = inflation_dataframe,
                     countries_dataframe = countries_dataframe)


# The cpiLookup function from Tutorial 4 doesn't work with our dataset, which shows repeated release years. The below function cannot be used with mutate() on such a dataset.

# cpiLookup <- function(y) { 
#                  return(CPITable %>% 
#                         filter(Year %in% y) %>%
#                         select(Value) %>% 
#                         .$Value ) }
\end{lstlisting}

\begin{lstlisting}
# Creating inflation-adjusted attributes

mov_df <- mov_df %>%
  mutate(adjustedRev = adjust_for_inflation(revenue, release_year,
                                            country, 2022,
                     inflation_dataframe = inflation_dataframe,
                     countries_dataframe = countries_dataframe))


mov_df <- mov_df %>%
  mutate(adjustedBudget = adjust_for_inflation(budget, release_year,
                                            country, 2022,
                     inflation_dataframe = inflation_dataframe,
                     countries_dataframe = countries_dataframe))

mov_df %>%
  select(release_year, budget, adjustedBudget)
\end{lstlisting}

\begin{lstlisting}
# After adjusting for inflation, which movies have the top revenues
mov_df %>%
  select(title, adjustedRev, release_year) %>%
  arrange(-adjustedRev)
\end{lstlisting}

\begin{lstlisting}
### Creating an attribute for the quartile of inflation-adjusted revenue
breaks_adjRev_qrts <- quantile(mov_df$adjustedRev, c(0.75))
breaks_adjRev_qrts

# Binning accordingly (but approximating the break values with even numbers that are easier for viewers to read.)
mov_df <- mov_df %>%
  mutate(adjRev_quartile = cut(adjustedRev,
                               breaks = c(0, 209953708, 10000000000),
                               labels = c("Lower 3 quartiles of revenue",
                                          "Top revenue quartile"),
                               include.lowest=TRUE))
\end{lstlisting}

\begin{lstlisting}
### Creating a *release season* attribute
# Defining a function getSeason (found on StackOverflow) and use it to create a release_season attribute from release_date.

getSeason <- function(input.date){
  numeric.date <- 100*month(input.date)+day(input.date)
  ## input Seasons upper limits in the form MMDD in the "break =" option:
  cuts <- base::cut(numeric.date, breaks = c(0,319,0620,0921,1220,1231)) 
  # rename the resulting groups (could've been done within cut(...levels=) if "Winter" wasn't double
  levels(cuts) <- c("Winter","Spring","Summer","Fall","Winter")
  return(cuts)
}

mov_df <- mov_df %>%
  mutate(release_season = getSeason(release_date))
\end{lstlisting}

\begin{lstlisting}
# Creating attribute that tracks whether the season of release was warmer or 
# cooler. Used to analyze how some genres thrive in warmer months.

mov_df <- mov_df %>%
  mutate(warmSzn_movie = ifelse(release_season %in% c("Spring", "Summer"), 
                                paste("Warmer Season (", PrimaryGenre, ")"), 
                                paste("Cooler Season (", PrimaryGenre, ")")))

head(mov_df, 10)

# Writing this preprocessed data to file
write.csv(mov_df, "C:\\Users\\maxru\\OneDrive\\Documents\\DePaul MS Work\\DSC 465\\Homework\\HW_4\\mov_df_processed.csv", row.names = FALSE)
\end{lstlisting}

\begin{lstlisting}
# Prepping data for Sankey, specifically
# Including only film records that show a PrimaryGenre that is top-selling

# Identifying the Genres that earned the most revenue in this period
mov_df %>%
  dplyr::group_by(PrimaryGenre) %>%
  dplyr::summarise(total_rev = sum(adjustedRev)) %>%
  arrange(-total_rev) %>%
  head(10)
\end{lstlisting}

\begin{lstlisting}
# Improved Seasonal Sankey
# Narrowing dataset to include only records associated with a top-5-earning genre.
summer_movie_sankey_df <- mov_df %>%
  filter(PrimaryGenre %in% c("Family", "Science Fiction", "Adventure", "Animation"))
\end{lstlisting}

\begin{lstlisting}
# TESTING WARMER COOLER SEASON
summer_movie_sankey_df <- summer_movie_sankey_df %>%
  select(PrimaryGenre, warmSzn_movie, adjRev_quartile)


summer_movie_Sankey <- hchart(data_to_sankey(summer_movie_sankey_df), "sankey", name = "Relating season of release, genre, and adj-revenue quartile")

summer_movie_Sankey <- summer_movie_Sankey %>%
  hc_title(text = "Seasonal blockbusting rate is distinct from seasonal average") %>%
  hc_subtitle(text = "Though these genres might all register higher average revenues during the warmer months, their proportions of blockbusters in that time don't necessarily improve.")

summer_movie_Sankey
\end{lstlisting}

\begin{lstlisting}
# Saving the sankey HTML and png files
htmlwidgets::saveWidget(widget = summer_movie_Sankey, file = 'C:\\Users\\maxru\\OneDrive\\Documents\\DePaul MS Work\\DSC 465\\Homework\\HW_4\\summer_blockbuster_sankey.html')


webshot::install_phantomjs()

webshot::webshot(url = "C:\\Users\\maxru\\OneDrive\\Documents\\DePaul MS Work\\DSC 465\\Homework\\HW_4\\summer_blockbuster_sankey.html",
               file = "C:\\Users\\maxru\\OneDrive\\Documents\\DePaul MS Work\\DSC 465\\Homework\\HW_4\\summer_blockbuster_sankey.png" , delay = 5)
\end{lstlisting}

\begin{lstlisting}
## SUMMER MOVIE SHARE CHART

# Summer genre count shares
mov_summer_df <- mov_df

mov_summer_df %>%
  filter(release_season %in% c("Spring", "Summer"))

mov_summer_df <- mov_summer_df %>%
  mutate(summerGenre = ifelse(PrimaryGenre %in% c("Adventure", "Family",
                                           "Science Fiction", "Animation", "Thriller"),
                       PrimaryGenre,
                       "Other"))

mov_summer_df$summerGenre <- factor(mov_summer_df$summerGenre, 
                                       levels = c("Adventure", "Animation",
                                                  "Science Fiction", "Family",
                                                  "Thriller", "Other"))

summer_counts_df <- mov_summer_df %>%
  count(summerGenre) %>%
  arrange(-n)

# summer_counts_df

movie_count <- nrow(mov_summer_df)

summer_counts_df <- summer_counts_df %>%
  mutate(count_share = n / movie_count)

summer_counts_df
\end{lstlisting}

\begin{lstlisting}
# Summer genre revenue shares
summer_revs_df <- mov_summer_df %>%
  group_by(summerGenre) %>%
  summarise(tot_adjRev = sum(adjustedRev)) %>%
  arrange(-tot_adjRev)

summer_adjRev_total = sum(summer_revs_df$tot_adjRev)

summer_revs_df <- summer_revs_df %>%
  mutate(rev_share = tot_adjRev / summer_adjRev_total)

summer_revs_df
\end{lstlisting}

\begin{lstlisting}
# Summer genre popularity shares
summer_pops_df <- mov_summer_df %>%
  group_by(summerGenre) %>%
  summarise(tot_pop = sum(popularity)) %>%
  arrange(-tot_pop)

summer_pop_total_all = sum(summer_pops_df$tot_pop)

summer_pops_df <- summer_pops_df %>%
  mutate(pop_share = tot_pop / summer_pop_total_all)

summer_pops_df
\end{lstlisting}

\begin{lstlisting}
# joining these aggregations, message = FALSE
summer_mov_proport_plot_df <- summer_counts_df %>%
  inner_join(summer_revs_df) %>%
  inner_join(summer_pops_df) %>%
  select(summerGenre, count_share, rev_share, pop_share)

summer_mov_proport_plot_df

summer_mov_proport_plot_df <- summer_mov_proport_plot_df %>%
  pivot_longer(!summerGenre, names_to = "Film_Measure", values_to ="Share")

summer_mov_proport_plot_df <- summer_mov_proport_plot_df %>%
  mutate(Film_Measure = fct_recode(as.factor(Film_Measure),
                                   "Sample Count" = "count_share",
                                   "Revenue" = "rev_share",
                                   "Popularity" = "pop_share"))




levels(summer_mov_proport_plot_df$Film_Measure) <- c("Sample Count", "Revenue", "Popularity")
levels(summer_mov_proport_plot_df$Film_Measure)


summer_mov_proport_plot_df <- summer_mov_proport_plot_df %>%
  mutate(Share_rnd = round(Share, 3))

summer_mov_proport_plot_df
\end{lstlisting}

\begin{lstlisting}
# FINAL VIZ
proport_plot_palette <-  c("Action"="#94aeca", "Adventure"="#f7ba7e",
                   "Comedy"="#cfaec6", "Drama"="#ed999a", "Animation"="#acd3d0",
                   "Other"="#d5cfcd", "Thriller"="#5a5858", "Science Fiction"="#B9fdbe", "Family"="#Cfabfb")

summer_mov_proport_plot_df %>%
  filter(summerGenre != "Other") %>%
  ggplot(aes(x = summerGenre, y = Share, fill = summerGenre)) + 
  geom_bar(stat='identity') + 
facet_wrap(~Film_Measure) +
  geom_text(aes(label = Share_rnd),
            color = "white", size = 4.5,
            vjust = 1, fontface = "bold")  +
  scale_y_continuous(labels = scales::percent) + 
  scale_fill_manual(name = "Genre", values = proport_plot_palette) +
  guides(fill = guide_legend(ncol = 5)) +
  ggtitle("Some Genres Thrive in Warmer Weather") +
  theme_minimal() +
  theme(plot.title = element_text(hjust = 0.5, size = 20, 
                                  vjust = 2, face = "bold"),
        plot.subtitle = element_text(hjust = 0.5),
        legend.position="bottom",
        legend.box = "horizontal",
        legend.title = element_text(size = 16),
        legend.text = element_text(size = 12),
        axis.title.x=element_blank(),
        axis.text.x=element_blank(),
        axis.ticks.x=element_blank(),
        strip.text = element_text(size = 16),
        panel.grid.major.x = element_blank()) +
  labs(subtitle = "Shares by genre of various totals from Spring and Summer releases, years 2001 to 2015.")
\end{lstlisting}

\begin{lstlisting}
# Making the (previous) Sankey

# Narrowing dataset to include only records associated with a top-5-earning genre.
mov_df_top_gens <- mov_df %>%
  filter(PrimaryGenre %in% c("Action", "Adventure", "Drama", "Comedy", "Animation"))

# TESTING WARMER COOLER SEASON
season_genre_sankey_df <- mov_df_top_gens %>%
  select(PrimaryGenre, warmSzn_movie, adjRev_quartile)


season_genre_Sankey <- hchart(data_to_sankey(season_genre_sankey_df), "sankey", name = "Relating season of release, genre, and adj-revenue quartile")

season_genre_Sankey <- season_genre_Sankey %>%
  hc_title(text = "Season of Release, Genre, and Revenue Quartile") %>%
  hc_subtitle(text = "Examining how release season, genre, and revenue quartile relate in terms of distribution. Revenues were inflation-adjusted to their 2022 equivalents.")

season_genre_Sankey
\end{lstlisting}

\begin{lstlisting}
season_genre_sankey_df <- mov_df_top_gens %>%
  select(release_season, PrimaryGenre, adjRev_quartile)


season_genre_Sankey <- hchart(data_to_sankey(season_genre_sankey_df), "sankey", name = "Relating season of release, genre, and adj-revenue quartile")

season_genre_Sankey <- season_genre_Sankey %>%
  hc_title(text = "Season of Release, Genre, and Revenue Quartile") %>%
  hc_subtitle(text = "Examining how release season, genre, and revenue quartile relate in terms of distribution. Revenues were inflation-adjusted to their 2022 equivalents.")

season_genre_Sankey
\end{lstlisting}

\begin{lstlisting}
# Saving the sankey HTML and png files
htmlwidgets::saveWidget(widget = season_genre_Sankey, file = 'C:\\Users\\maxru\\OneDrive\\Documents\\DePaul MS Work\\DSC 465\\Homework\\HW_4\\season-genre-adjRevQ_sankey_all_years.html')


# webshot::install_phantomjs()

webshot::webshot(url = "C:\\Users\\maxru\\OneDrive\\Documents\\DePaul MS Work\\DSC 465\\Homework\\HW_4\\season-genre-adjRevQ_sankey_all_years.html",
                 file = "C:\\Users\\maxru\\OneDrive\\Documents\\DePaul MS Work\\DSC 465\\Homework\\HW_4\\season-genre-adjRevQ_sankey_all_years.png" , delay = 5)
\end{lstlisting}

\begin{lstlisting}
# Original Proportionality-by-Genre Bar Chart (of the 5 top-grossing genres)

# proportionality by genre
mov_df2 <- mov_df

# head(mov_df2)

mov_df2 <- mov_df2 %>%
  mutate(bigGenre = ifelse(PrimaryGenre %in% c("Action", "Adventure", "Drama",
                                           "Comedy", "Animation"),
                       PrimaryGenre,
                       "Other"))

mov_df2_gen_cnts <- mov_df2 %>%
  count(bigGenre) %>%
  arrange(-n)

movie_count = nrow(mov_df2)

mov_df2_gen_cnts <- mov_df2_gen_cnts %>%
  mutate(count_share = n / movie_count)

mov_df2_gen_cnts
\end{lstlisting}

\begin{lstlisting}
# Revenue shares by genre
mov_df2_rev_sums <- mov_df2 %>%
  group_by(bigGenre) %>%
  summarise(tot_adjRev = sum(adjustedRev)) %>%
  arrange(-tot_adjRev)

adjRev_total = sum(mov_df2_rev_sums$tot_adjRev)

mov_df2_rev_sums <- mov_df2_rev_sums %>%
  mutate(rev_share = tot_adjRev / adjRev_total)

mov_df2_rev_sums
\end{lstlisting}

\begin{lstlisting}
# Popularity shares by genre
mov_df2_pop_sums <- mov_df2 %>%
  group_by(bigGenre) %>%
  summarise(tot_pop = sum(popularity)) %>%
  arrange(-tot_pop)

pop_total_all = sum(mov_df2_pop_sums$tot_pop)

mov_df2_pop_sums <- mov_df2_pop_sums %>%
  mutate(pop_share = tot_pop / pop_total_all)

mov_df2_pop_sums
\end{lstlisting}

\begin{lstlisting}
# adjBudget shares by genre
mov_df2_budg_sums <- mov_df2 %>%
  group_by(bigGenre) %>%
  summarise(tot_adjBudg = sum(adjustedBudget)) %>%
  arrange(-tot_adjBudg)

adjBudg_total_all = sum(mov_df2_budg_sums$tot_adjBudg)

mov_df2_budg_sums <- mov_df2_budg_sums %>%
  mutate(budg_share = tot_adjBudg / adjBudg_total_all)

mov_df2_budg_sums
\end{lstlisting}

\begin{lstlisting}
# joining these aggregations, message = FALSE
mov_df_proportional_plot_df <- mov_df2_gen_cnts %>%
  inner_join(mov_df2_rev_sums) %>%
  inner_join(mov_df2_pop_sums) %>%
  inner_join(mov_df2_budg_sums) %>%
  select(bigGenre, count_share, rev_share, pop_share, budg_share)
\end{lstlisting}

\begin{lstlisting}
# pivoting-longer these joined tables
mov_df_proportional_plot_df <- mov_df_proportional_plot_df %>%
  pivot_longer(!bigGenre, names_to = "Film_Measure", values_to ="Share")

mov_df_proportional_plot_df
\end{lstlisting}

\begin{lstlisting}
## Visualizing shares

# Rough sketch of genre proportions across the various measures
mov_df_proportional_plot_df %>%
  ggplot(aes(x = bigGenre, y = Share)) + geom_bar(stat='identity') + facet_wrap(~Film_Measure)
\end{lstlisting}

\begin{lstlisting}
# Linear correlations between Revenue and Popularity, by Genre

# Final Viz: Linear corrs by genre
corrsByGenre_PopVsRev <- mov_df2 %>%
  ggplot(aes(x = adjustedRev, y = popularity, color = bigGenre)) +
  geom_point(alpha = 0.1) +
  geom_smooth(method = lm, se = FALSE, color = "black", size = 0.2) +
  stat_cor(p.accuracy = 0.001, r.accuracy = 0.01,
           inherit.aes = FALSE, aes(x = adjustedRev, y = popularity)) +
  facet_wrap(~bigGenre) +
  scale_x_continuous(breaks = 500000000 * (0:4), labels = c("0", "0.5", "1", "1.5", "2"),
                     limits = c(0, 2000000000)) +
  scale_y_continuous(limits = c(0, 320)) +
  guides(color = FALSE) +
  labs(x = "Revenue ($billions)",
       y = "Popularity (TMDB Score)",
       title = "Correlations between Revenue and Popularity by Genre")

corrsByGenre_PopVsRev
\end{lstlisting}

\begin{lstlisting}
# RECOLORING: Linear corrs by genre
# Important Animation genre has been too lightly colored in this scheme. But 
# I've already recolored all the other visualizations...  :(

corrsByGenre_PopVsRev <- mov_df2 %>%
  ggplot(aes(x = adjustedRev, y = popularity, color = bigGenre)) +
  geom_point(alpha = 0.18) +
  geom_smooth(method = lm, se = FALSE, color = "black", size = 0.2) +
  stat_cor(p.accuracy = 0.001, r.accuracy = 0.01,
           inherit.aes = FALSE, aes(x = adjustedRev, y = popularity)) +
  facet_wrap(~bigGenre) +
  scale_color_manual(name = "Genre", values = top5_palette) +
  scale_x_continuous(breaks = 500000000 * (0:4), labels = c("0", "0.5", "1", "1.5", "2"),
                     limits = c(0, 2000000000)) +
  scale_y_continuous(limits = c(0, 320)) +
  guides(color = FALSE) +
  labs(x = "Revenue ($billions)",
       y = "Popularity (TMDB Score)",
       title = "Correlations between Revenue and Popularity by Genre")

corrsByGenre_PopVsRev
\end{lstlisting}

\begin{lstlisting}
# Popularity versus Revenue overall, and its density

## This is a two-part visualization

Part 1 - Popularity vs Revenue overall
pop_v_adjRev <- mov_df2 %>%
  ggplot(aes(x = adjustedRev, y = popularity)) +
  geom_point(alpha = 0.2) +
  annotate("rect", xmin = 0, xmax = 300000000, ymin = 0, ymax = 125,
           color = "cyan", fill = NA) +
  geom_smooth(method = lm, se = FALSE, color = "red", size = 0.2) +
  stat_cor(p.accuracy = 0.001, r.accuracy = 0.01,
           inherit.aes = FALSE, aes(x = adjustedRev, y = popularity), color = "red") +
  scale_x_continuous(labels = label_comma()) + 
  labs(title = "Popularity vs Revenue",
       x = "Revenue ($)",
       y = "Popularity (TMDB Score)")

pop_v_adjRev

# The light blue box indicates what area will be featured in the 2D-density plot
# that follows.
\end{lstlisting}

\begin{lstlisting}
# Part 2 - "Zooming in" with a 2D Density Plot
pop_v_adjRev_zoomedUp <- mov_df2 %>%
  ggplot(aes(x = adjustedRev, y = popularity)) +
  stat_density2d() +
  geom_point(alpha = 0.08) +
  annotate("rect", xmin = 0, xmax = 300000000, ymin = 0, ymax = 125,
           color = "cyan", fill = NA) +
  # scale_x_continuous(labels = label_comma())
  scale_x_continuous(limits = c(0, 300000000),
                     labels = label_comma()) +
  scale_y_continuous(limits = c(0, 125)) + 
  labs(title = "Zooming in on Popularity vs Revenue - A Density Plot",
       x = "Revenue ($)",
       y = "Popularity (TMDB Score)")

pop_v_adjRev_zoomedUp
\end{lstlisting}

\begin{lstlisting}
# Final Viz: Popularity versus Revenue overall, and its density
comboPlot_popVsRev <- ggarrange(pop_v_adjRev, pop_v_adjRev_zoomedUp, nrow = 2)

comboPlot_popVsRev
\end{lstlisting}

\begin{lstlisting}
# Stacked Area Chart Redo
# Creating a dedicated subset of data
# Creating a df specific to this viz

stackedArea_df <- mov_df2

# Setting levels to my Genre field, so that the desired order will be reflected
# by the area chart.

# mov_df2_rev_sums

stackedArea_df$bigGenre <- factor(stackedArea_df$bigGenre,
                                  levels = c("Action", "Adventure", "Comedy",
                                             "Drama", "Animation", "Other"))
# Cutting df so that the oldest year is 2001

stackedArea_df <- stackedArea_df %>%
  filter(release_year >= 2001)
\end{lstlisting}

\begin{lstlisting}
stackedArea_df2 <- stackedArea_df %>%
  group_by(release_year, bigGenre) %>%
  summarise(adjRevSum = sum(adjustedRev)) %>%
  select(release_year, bigGenre, adjRevSum) %>%
  arrange(release_year)
\end{lstlisting}

\begin{lstlisting}
# Recoloring and reordering area chart
top5_palette <-  c("Action"="#94aeca", "Adventure"="#f7ba7e",
                   "Comedy"="#cfaec6", "Drama"="#ed999a", "Animation"="#acd3d0",
                   "Other"="#d5cfcd")

plot_stackedArea_revByGen <- stackedArea_df2 %>%
  ggplot(aes(x = release_year, y = adjRevSum, fill = bigGenre)) +
  geom_area() +
  scale_fill_manual(values = top5_palette, name = "Genre") +
  labs(title = "Total Revenue of Top-Grossing Genres (2001-2015)",
       y = "Revenue ($ billion)",
       x = "Year") +
  scale_y_continuous(breaks = 10000000000 * (0:3), labels = c("0", "10", "20", "30")) +
  scale_x_continuous(breaks = seq(1, 2015, by=2)) +
  theme_minimal() +
  theme(panel.grid.major.x = element_blank(),
        panel.grid.minor.x = element_blank())

plot_stackedArea_revByGen
\end{lstlisting}

\begin{lstlisting}
# Previous attempts and scratch visualizations

# Final visualization of genre proportions across various measures
mov_df_proportional_plot_df$bigGenre <-
  factor(mov_df_proportional_plot_df$bigGenre,
         levels = c('Other', 'Action', 'Adventure', 'Comedy', 'Drama', 'Animation'))

mov_df_proportional_plot_df <- mov_df_proportional_plot_df %>%
  mutate(Share_rnd = round(Share, 2))

mov_df_proportional_plot_df %>%
  filter(Film_Measure != "budg_share") %>%
  ggplot(aes(fill = bigGenre, x = bigGenre, y = Share)) +
  geom_bar(position="dodge", stat='identity') +
  geom_text(aes(label = Share_rnd),
            color = "white", size = 4.5,
            vjust = 1.5, fontface = "bold") +
  facet_wrap(~factor(Film_Measure,
                     levels = c('count_share', 'rev_share', 'pop_share'),
                     labels = c('Sample Count', 'Revenue', 'Popularity'))) +
  scale_y_continuous(labels = scales::percent,
                     limits = c(0, 0.3)) +
  scale_x_discrete(guide = guide_axis(n.dodge = 3)) + 
  scale_fill_brewer(name = "Genre", palette = "Set2") +
  guides(fill = guide_legend(ncol = 6)) +
  ggtitle("Genre Proportions of Several Measures") +
  theme_minimal() +
  theme(plot.title = element_text(hjust = 0.5, size = 20, 
                                  vjust = 2, face = "bold"),
        legend.position="bottom",
        legend.box = "horizontal",
        legend.title = element_text(size = 18),
        legend.text = element_text(size = 13),
        axis.title.x=element_blank(),
        axis.text.x=element_blank(),
        axis.ticks.x=element_blank(),
        strip.text = element_text(size = 16),
        panel.grid.major.x = element_blank())
\end{lstlisting}

\begin{lstlisting}
# 2D-Density plot specific to genre
# I was coding this to be iterated through for each genre, eventually arranging
# them together as six plots in one image. However, I decided against it,
# because I thought it more informative to keep all the axes consistent (as 
# facet_wrap does automatically.) 

the_var = "Action"
mov_df2 %>%
  filter(bigGenre == the_var) %>%
  
  ggplot(aes(x = adjustedRev, y = popularity)) +
  stat_density2d(aes(colour=..level..)) +
  scale_x_continuous(labels = label_comma()) +
  labs(title = the_var) +
  theme(plot.title = element_text(hjust = 0.5, size = 14))
\end{lstlisting}

\begin{lstlisting}
# sketching "corr by genre" viz: manually zooming in, away from outliers
mov_df2 %>%
  ggplot(aes(x = adjustedRev, y = popularity, color = bigGenre, alpha = 0.05)) +
  geom_point() +
  facet_wrap(~bigGenre) +
  scale_x_continuous(limits = c(0, 2000000000)) +
  scale_y_continuous(limits = c(0, 320))
\end{lstlisting}

\begin{lstlisting}
# sketching "corr by genre" viz: Adding regression lines, final polishes
mov_df2 %>%
  # ggplot(aes(x = adjustedRev, y = popularity, color = bigGenre, alpha = 0.05)) +
  # geom_point() +
  ggplot(aes(x = adjustedRev, y = popularity, color = bigGenre)) +
  geom_point(alpha = 0.1) +
  geom_smooth(method = lm, se = FALSE, color = "black", size = 0.2) +
  stat_cor(p.accuracy = 0.001, r.accuracy = 0.01,
           inherit.aes = FALSE, aes(x = adjustedRev, y = popularity)) +
  facet_wrap(~bigGenre) +
  scale_x_continuous(breaks = 500000000 * (0:4), labels = c("0", "0.5", "1", "1.5", "2"),
                     limits = c(0, 2000000000)) +
  scale_y_continuous(limits = c(0, 320)) +
  guides(color = FALSE) +
  labs(x = "Revenue ($billions)",
       y = "Popularity",
       title = "Correlations between Revenue and Popularity by Genre")
\end{lstlisting}

\begin{lstlisting}
## Desparate attempts at faceted density plots
Part 1 - Levels only
mov_df2 %>%
  ggplot(aes(x = adjustedRev, y = popularity)) +
  stat_density2d(aes(colour=..level..)) +
  facet_wrap(~bigGenre) +
  scale_x_continuous(limits = c(0, 2000000000)) +
  scale_y_continuous(limits = c(0, 320))

Part 2 - Levels, with faint points
mov_df2 %>%
  ggplot(aes(x = adjustedRev, y = adjustedBudget)) +
  stat_density2d(aes(colour=..level..)) +
  geom_point(color = "yellow", alpha = 0.02) +
  scale_x_continuous(labels = label_comma()) +
  scale_y_continuous(labels = label_comma()) +
  facet_wrap(~bigGenre)

Part 3 - Alpha tile-densities, plus points
mov_df2 %>%
  ggplot(aes(x = adjustedRev, y = popularity)) + 
  geom_point(aes(color = bigGenre)) +
  stat_density2d(aes(alpha=..density..),
                 geom="tile",
                 contour=FALSE) +
  facet_wrap(~bigGenre) +
  scale_x_continuous(limits = c(0, 2000000000)) +
  scale_y_continuous(limits = c(0, 320))
\end{lstlisting}

\begin{lstlisting}
## Early Panel Plot

# Horizontal panel plot: Popularity vs Revenue by Genre
# I found it too wide for the popularity values to be decoded meaningfully, even
# after removing several outliers of adjusted revenue. 

mov_df_top_gens %>%
  ggplot(aes(x = adjustedRev, y = popularity)) + geom_point() + facet_grid(PrimaryGenre ~ .)


mov_df_top_gens %>%
  filter(adjustedRev < 2000000000) %>%
  ggplot(aes(x = adjustedRev, y = popularity)) + geom_point() + facet_grid(PrimaryGenre ~ .)
\end{lstlisting}

\begin{lstlisting}
# Sankey drafts
## Genre, Revenue Quartile, and Top 10 Production Companies
### Note that this uses raw revenue values (without inflation adjustment.)
### This work was done before inflation-adjustment was implemented.

# Identifying prod. companies with top revenue sums
# Identifying the 10 highest revenue-generating Production Companies
top_10_prod_companies <- mov_df_top_gens %>%
  dplyr::group_by(PrimaryProductionCompany) %>%
  dplyr::summarise(tot_revenue = sum(revenue)) %>%
  arrange(-tot_revenue) %>%
  select(PrimaryProductionCompany) %>%
  head(10)

top_10_prod_companies$PrimaryProductionCompany


# Narrowing dataset to include only records associated with these top-10-earning
# production companies.
mov_top_gens_and_prodCos <- mov_df_top_gens %>%
  filter(PrimaryProductionCompany %in% top_10_prod_companies$PrimaryProductionCompany)
\end{lstlisting}

\begin{lstlisting}
# Binning revenue into quartiles - identifying break points
genre_prodco_sankey_df <- mov_top_gens_and_prodCos %>%
  select(PrimaryGenre, revenue, PrimaryProductionCompany)

breaks_rev_qrts <- quantile(genre_prodco_sankey_df$revenue, c(0.25 * (0:4)))
breaks_rev_qrts
\end{lstlisting}

\begin{lstlisting}
# Binning revenue into quartiles - creating "revenueQuartile" attribute
# Binning accordingly (somewhat roughly, so that my break values will be 
# prettier to the eye)
genre_prodco_sankey_df <- genre_prodco_sankey_df %>%
  mutate(revenueQuartile = cut(revenue,
                               breaks = c(0, 41000000, 101000000, 
                                          217000000, 1850000000),
                               labels = c("$0 - $41,000,000",
                                          "$41,000,000 - $101,000,000",
                                          "$101,000,000 - $217,000,000",
                                          "$217,000,000 - $1,850,000,000"),
                               include.lowest=TRUE)) %>%
  select(PrimaryGenre, revenueQuartile, PrimaryProductionCompany)
\end{lstlisting}

\begin{lstlisting}
# Creating Sankey of Genre, Rev Quartile, and Prod Co.
# genre_prodco_sankey <- hchart(data_to_sankey(genre_prodco_sankey_df), "sankey", name = "Relating revenue to genre and production company")
# 
# genre_prodco_sankey <- genre_prodco_sankey %>%
#   hc_title(text = "Revenue outcomes based on Genre and Production Company") %>%
#   hc_subtitle(text = "Examining the distributions of genre and production 
#               company for each quartile of revenue earned.")
# 
# genre_prodco_sankey

sankeySeasons_df <- mov_top_gens_and_prodCos %>%
  mutate(revenueQuartile = cut(revenue,
                               breaks = c(0, 41000000, 101000000, 
                                          217000000, 1850000000),
                               labels = c("$0 - $41,000,000",
                                          "$41,000,000 - $101,000,000",
                                          "$101,000,000 - $217,000,000",
                                          "$217,000,000 - $1,850,000,000"),
                               include.lowest=TRUE)) %>%
  select(PrimaryGenre, release_season, revenueQuartile)
\end{lstlisting}

\begin{lstlisting}
seasonSankey <- hchart(data_to_sankey(sankeySeasons_df), "sankey", name = "Relating revenue to genre and season of release")

seasonSankey <- seasonSankey %>%
  hc_title(text = "Revenue outcomes based on Genre and Release Season") %>%
  hc_subtitle(text = "Examining the distributions of genre and release season for each quartile of revenue earned.")

seasonSankey

# htmlwidgets::saveWidget(widget = seasonSankey, file = 'C:\\Users\\maxru\\OneDrive\\Documents\\DePaul MS Work\\DSC 465\\Homework\\HW_4\\genre-season_sankey.html')
# # webshot::install_phantomjs()
# webshot::webshot(url = "C:\\Users\\maxru\\OneDrive\\Documents\\DePaul MS Work\\DSC 465\\Homework\\HW_4\\genre-season_sankey.html",
#                  file = "C:\\Users\\maxru\\OneDrive\\Documents\\DePaul MS Work\\DSC 465\\Homework\\HW_4\\genre-season_sankey.png" , delay = 5)
# # knitr::include_graphics("genre-season_sankey.png")
\end{lstlisting}

\begin{lstlisting}
season_genre_sankey_df <- mov_top_gens_and_prodCos %>%
  mutate(revenueQuartile = cut(revenue,
                               breaks = c(0, 41000000, 101000000, 
                                          217000000, 1850000000),
                               labels = c("$0 - $41,000,000",
                                          "$41,000,000 - $101,000,000",
                                          "$101,000,000 - $217,000,000",
                                          "$217,000,000 - $1,850,000,000"),
                               include.lowest=TRUE)) %>%
  select(release_season, PrimaryGenre, revenueQuartile)
\end{lstlisting}

\begin{lstlisting}
season_genre_Sankey <- hchart(data_to_sankey(season_genre_sankey_df), "sankey", name = "Relating revenue to genre and season of release")

season_genre_Sankey <- season_genre_Sankey %>%
  hc_title(text = "Season of Release, Genre, and Revenue Quartile") %>%
  hc_subtitle(text = "Examining how release season, genre, and revenue quartile relate in terms of distribution.")

season_genre_Sankey

# htmlwidgets::saveWidget(widget = season_genre_Sankey, file = 'C:\\Users\\maxru\\OneDrive\\Documents\\DePaul MS Work\\DSC 465\\Homework\\HW_4\\season-genre-revQ_sankey.html')
# 
# 
# # webshot::install_phantomjs()
# 
# webshot::webshot(url = "C:\\Users\\maxru\\OneDrive\\Documents\\DePaul MS Work\\DSC 465\\Homework\\HW_4\\season-genre-revQ_sankey.html",
#                  file = "C:\\Users\\maxru\\OneDrive\\Documents\\DePaul MS Work\\DSC 465\\Homework\\HW_4\\season-genre-revQ_sankey.png" , delay = 5)
# 
# knitr::include_graphics("genre-season_sankey.png")
\end{lstlisting}