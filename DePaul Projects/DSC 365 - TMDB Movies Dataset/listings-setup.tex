% \usepackage{listings}
% \usepackage{xcolor}
% % Define custom colors
% \definecolor{codegreen}{rgb}{0,0.6,0}
% \definecolor{codegray}{rgb}{0.5,0.5,0.5}
% \definecolor{codepurple}{rgb}{0.58,0,0.82}
% \definecolor{codeorange}{rgb}{1,0.49,0}
% \definecolor{backcolour}{rgb}{0.95,0.95,0.92}


% \lstset{language=Python,                 % the language of the code
%   basicstyle=\ttfamily\small,       % the size of the fonts that are used for the code
%   numbers=left,                    % where to put the line-numbers
%   numberstyle=\tiny\color{gray},   % the style that is used for the line-numbers
%   stepnumber=1,                    % the step between two line-numbers. If it is 1, each line will be numbered
%   numbersep=5pt,                   % how far the line-numbers are from the code
%   backgroundcolor=\color{backcolour},   % choose the background color. You must add \usepackage{color}
%   showspaces=false,                % show spaces adding particular underscores
%   showstringspaces=false,          % underline spaces within strings
%   showtabs=false,                  % show tabs within strings adding particular underscores
%   frame=single,                    % adds a frame around the code
%   rulecolor=\color{black},         % if not set, the frame-color may be changed on line-breaks within not-black text (e.g. comments (green here))
%   tabsize=2,                       % sets default tabsize to 2 spaces
%   captionpos=b,                    % sets the caption-position to bottom
%   breaklines=true,                 % sets automatic line breaking
%   breakatwhitespace=false,         % sets if automatic breaks should only happen at whitespace
%   title=\lstname,                  % show the filename of files included with \lstinputlisting;
%                                   % also try caption instead of title
%   keywordstyle=\color{blue},       % keyword style
%   commentstyle=\color{dgreen},     % comment style
%   stringstyle=\color{codepurple},       % string literal style
%   escapeinside={\%*}{*)},          % if you want to add LaTeX within your code
%   morekeywords={*,...}             % if you want to add more keywords to the set
% }

\usepackage{listings}
\usepackage{xcolor}
% Define custom colors
\definecolor{codegreen}{rgb}{0,0.6,0}
\definecolor{codegray}{rgb}{0.5,0.5,0.5}
\definecolor{codepurple}{rgb}{0.58,0,0.82}
\definecolor{codeorange}{rgb}{1,0.49,0}
\definecolor{backcolour}{rgb}{0.95,0.95,0.92}

% Global settings
\lstset{
  basicstyle=\ttfamily\small,
  numbers=left,
  numberstyle=\tiny\color{codegray},
  stepnumber=1,
  numbersep=5pt,
  backgroundcolor=\color{backcolour},
  showspaces=false,
  showstringspaces=false,
  showtabs=false,
  frame=single,
  rulecolor=\color{black},
  tabsize=2,
  captionpos=b,
  breaklines=true,
  breakatwhitespace=false,
  title=\lstname,
  escapeinside={\%*}{*)},
}

% Python style
\lstdefinestyle{pythonstyle}{
  language=Python,
  keywordstyle=\color{blue},
  commentstyle=\color{codegreen},
  stringstyle=\color{codepurple},
  morekeywords={*,...}
}

% R style
\lstdefinestyle{rstyle}{
  language=R,
  keywordstyle=\color{blue},
  commentstyle=\color{codegreen},
  stringstyle=\color{codepurple},
  morekeywords={*,...},
  alsoletter={.}
}